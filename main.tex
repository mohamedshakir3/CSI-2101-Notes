\documentclass[openany]{report}
\usepackage[utf8]{inputenc}

\usepackage{lecture_notes_styles}
\usepackage{stylesheet}

\newtheorem{theorem}{Theorem}[section]
\newtheorem{definition}{Definition}[section]
\newtheorem{fact}{Fact}[section]
\newtheorem{prop}{Proposition}[section]
\newtheorem{corollary}{Corollary}[section]
\newtheorem{lemma}{Lemma}[section]

\title{CSI 2101 Lecture Notes}
\author{Mohamed Shakir}

\begin{document}

\maketitle

\tableofcontents
\chapter*{Definitions, Theorems, Lemmas, and Corollaries}
\label{chapter:theorems}
\addcontentsline{toc}{chapter}{\hyperref[chapter:theorems]{Definitions, Theorems, Lemmas, and Corollaries}}
\begin{manualdefinition}{\hyperref[definition4.1.1]{4.1.1}}
    Let $a$ and $b$ be two integers such that $a \neq 0$. We say that $a$ divides $b$ if there exists $c$ such that $b = ac$. If $a$ divides be we say $a$ is a factor or divisor of $b$. We also can say b is a multiple of a.

\end{manualdefinition}
\begin{manualtheorem}{\hyperref[theorem4.1.1]{4.1.1}}
    Let $a,b,c \in \ints$ with $a \neq 0$.
    \begin{enumerate}
        \item If $a\mid b$ and $a \mid c$, then $a \mid (b + c)$
        \item If $a \mid b$, then $a \mid bc$ for every integer 
        \item If $a \mid b$ and $b \mid c$, then $a \mid c$
    \end{enumerate}
\end{manualtheorem}
\begin{manualcorollary}{\hyperref[corollary4.1.1]{4.1.1}}
Let $a,b,c \in \ints$ with $a \neq 0$. If $a\mid b$ and $a \mid c$, $a \mid (mb + nc)$ for all integers $m$ and $n$
\end{manualcorollary}
\begin{manualtheorem}{\hyperref[theorem4.1.2]{4.1.2}}[The Division Algorithm]
    Let $a, d \in \ints$ with $d > 0$. There exists a unique $q$ and $r$ such that 
$$0 \leq r \leq d$$
and 
$$a = dq + r$$
We write 
$$q = a \ div \ d$$
$$r=  a \ mod \ a $$
\end{manualtheorem}

\begin{manualdefinition}{\hyperref[definition4.1.2]{4.1.2}}
        Let $a, b, m \in \ints$ with $m \geq 2$. We say $a$ is congruent to $b$ modulo $m$ if $m \mid (a-b)$. We write $a \equiv b \ (mod \ m)$
\end{manualdefinition}

\begin{manualtheorem}{\hyperref[theorem4.1.3]{4.1.3}}
        Let $a,b,c,d,m \in \ints$ with $m \geq 2$. If $a \equiv b \ (mod \ m)$ and $c \equiv d \ (mod \ m)$, then 
    \begin{enumerate}
        \item $a + c \equiv b + d \ (mod \ m)$
        \item $ac \equiv bd \ (mod \ m)$
    \end{enumerate}
\end{manualtheorem}

\begin{manualdefinition}{\hyperref[definition5.1.1]{5.1.1}}
        A positive integer $p$ is prime if it admits exactly two divisors.
\end{manualdefinition}

\begin{manualtheorem}{\hyperref[theorem5.1.1]{5.1.1}}[Fundamental Theorem of Arithmetic]
      All integers greater than 1 can be written as a product of prime numbers. This representation is unique if we write the prime numbers in non-decreasing order.
\end{manualtheorem}

\begin{manualtheorem}{\hyperref[theorem5.1.2]{5.1.2}}
     Let $n > 1$ be an integer. If $n$ is not prime, then $n$ has a prime divisor $p$ such that $p \leq \sqrt{n}$.
\end{manualtheorem}

\begin{manualcorollary}{\hyperref[corollary6.0.1]{6.0.1}}
    Let 
    $$a = p_1^{a_1} \cdot p_2^{a_2} \cdot \ldots \cdot p_k^{a_k}$$
    $$a = p_1^{b_1} \cdot p_2^{b_2} \cdot \ldots \cdot p_k^{b_k}$$
    Where $p_i$ is prime, $a_i \geq 0$ and $b_i \geq 0$, $1 \leq i \leq k$. Then 
    $$gcd(a,b) = p_1^{\min(a_1,b_1)} \cdot p_2^{\min(a_2,b_2)} \cdot \ldots \cdot p_k^{\min(a_k,b_k)}$$
    $$lcm(a,b) = p_1^{\max(a_1,b_1)} \cdot p_2^{\max(a_2,b_2)} \cdot \ldots \cdot p_k^{\max(a_k,b_k)}$$
    $$gcd(a,b) \cdot lcm(a,b) = ab$$
\end{manualcorollary}


\begin{manuallemma}{\hyperref[lemma6.0.1]{6.0.1}}
Let $a,b,q,r$ be integers such that 
    $$a = b\cdot q + r$$
    Then 
    $$gcd(a,b)=gcd(b,r)$$
\end{manuallemma}

\begin{manualdefinition}{\hyperref[definition6.0.1]{6.0.1}}[Euclidean Algorithm]
     $$x = a$$
    $$y = b$$
    while $y \neq 0$
    $$r = x \mod y$$
    $$x = y$$
    $$y = r$$
    return $x$
\end{manualdefinition}

\begin{manualtheorem}{\hyperref[theorem6.0.1]{6.0.1}}[B\'ezout]
     Let $a,b \in \ints$ be positive integers. There exists $s,t \in \ints$ such that 
    $$s \cdot a + t \cdot b = gcd(a,b)$$
\end{manualtheorem}


\begin{manuallemma}{\hyperref[lemma6.0.1]{6.0.1}}
Let $a,b,c \in \ints$ with $a \neq 0$. If $gcd(a,b) = 1$ and $a \mid (bc)$, then $a \mid c$.
\end{manuallemma}

\begin{manuallemma}{\hyperref[lemma8.0.1]{8.0.1}}
    Let $a,b,c \in \ints$, with $a \neq 0$. If $gcd(a,b) = 1$, and $a \mid (bc)$, then $a|c$.

\end{manuallemma}

\begin{manualtheorem}{\hyperref[theorem8.0.1]{8.0.1}}
        Let $a,b,c,m \in \ints$, with $m \geq 2$. Assume $ac \equiv bc \ (mod \ m)$ and $gcd(c,m) = 1$. Then $a \equiv b \ (mod \ m)$.
\end{manualtheorem}

\begin{manuallemma}{\hyperref[lemma8.0.2]{8.0.2}}
        Let $p$ be a prime number and $a_1, a_2,\dots,a_n \in \ints$.
    If $p \mid (a_1 \cdot a_2 \cdot \dots \cdot a_n)$, then there exists $1 \leq i \leq n$ such that $p \mid a_i$.
\end{manuallemma}

\begin{manualtheorem}{\hyperref[theorem8.0.2]{8.0.2}}
        Let $m \in \ints$ with $m \geq 2$ and let $a \in \ints_m$. The multiplicative inverse of $a \ (mod \ n)$ exists if and only if $gcd(a,m) = 1$. When it exists, the inverse of  $a \ (mod \ n)$ is unique.
\end{manualtheorem}

%%%%%%%%%%%%%%%%%% Lecture 1 %%%%%%%%%%%%%%%%%%%%%%%%%%%%%%%%%
\chapter{Logic and Proof Techniques}
TBC. 



%%%%%%%%%%%%%%%%%% Lecture 2 %%%%%%%%%%%%%%%%%%%%%%%%%%%%%%%%%
\chapter{Proof Examples}
TBC.

%%%%%%%%%%%%%%%%%% Lecture 3 %%%%%%%%%%%%%%%%%%%%%%%%%%%%%%%%%
\chapter{Proof by Induction and More Examples}



%%%%%%%%%%%%%%%%%% Lecture 4 %%%%%%%%%%%%%%%%%%%%%%%%%%%%%%%%%
\chapter{Intro to Number Theory}
\section{Divisibility}
\begin{definition}\label{definition4.1.1}
    Let $a$ and $b$ be two integers such that $a \neq 0$. We say that $a$ divides $b$ if there exists $c$ such that $b = ac$. If $a$ divides be we say $a$ is a factor or divisor of $b$. We also can say b is a multiple of a.
\end{definition}
\begin{theorem}\label{theorem4.1.1}
    Let $a,b,c \in \ints$ with $a \neq 0$.
    \begin{enumerate}
        \item If $a\mid b$ and $a \mid c$, then $a \mid (b + c)$
        \item If $a \mid b$, then $a \mid bc$ for every integer 
        \item If $a \mid b$ and $b \mid c$, then $a \mid c$
    \end{enumerate}
\end{theorem}
\begin{proof}
    \begin{enumerate}
        \item We have to prove if $a \mid b$ and $a\mid c$, then $a \mid (b + c)$
        Let $a,b,c \in \ints$ with $a \neq 0$. Assume that $a \mid b$ and $a \mid c$, then for some $k,l \in \ints$
        $$b = k \cdot a$$
        $$c = l \cdot a$$
        Thus, we have
        $$b + c = k \cdot a + l \cdot a  = a(k + l)$$
        So $a \mid (b+c)$
        \item We have to prove if $a \mid b$, $a \mid bc$ for every $c$. Let $a,b \in \ints$ with $a \neq 0$. Assume that $a \mid b$. Then for some $k \in \ints$,
        $$b = k \cdot a$$
        Let $c \in \ints$, so 
        $$bc = k \cdot a \cdot c = a \cdot (kc)$$
        Therefore, $a \mid bc$
        \item We have to prove if $a \mid b$ and $b \mid c$, then $a \mid c.$ Let $a,b,c \in \ints$ with $a \neq 0$. Assume $a \mid b$ and $b \mid c$. Then we have for some $k, l \in \ints$
        $$b = k \cdot a$$
        $$c = l \cdot b$$
        So, 
        $$c = l \cdot b = l \cdot (k \cdot a) = (lk)a$$
        Therefore $a \mid c$
    \end{enumerate}
\end{proof}

\begin{corollary}\label{corollary4.1.1}
Let $a,b,c \in \ints$ with $a \neq 0$. If $a\mid b$ and $a \mid c$, $a \mid (mb + nc)$ for all integers $m$ and $n$
\end{corollary}
\begin{proof}
    Let $a,b c \in \ints$ with $a \neq 0$. Assume $a \mid b$  $a \mid c$. By the previous theorem (part 2), we have $a \mid mb$ and $a \mid nc$. Therefore, by the previous theorem (part 1), $a \mid (mb + nc)$
\end{proof}
\begin{theorem}[The Division Algorithm]\label{theorem4.1.2}
Let $a, d \in \ints$ with $d > 0$. There exists a unique $q$ and $r$ such that 
$$0 \leq r \leq d$$
and 
$$a = dq + r$$
We write 
$$q = a \ div \ d$$
$$r=  a \ mod \ a $$
\end{theorem}
\begin{definition}\label{definition4.1.2}
    Let $a, b, m \in \ints$ with $m \geq 2$. We say $a$ is congruent to $b$ modulo $m$ if $m \mid (a-b)$. We write $a \equiv b \ (mod \ m)$
\end{definition}
\textbf{Example:} Prove or disprove. We have $a \equiv b \ (mod \ m)$ if and only if $b \equiv a \ (mod \ m)$
\begin{align*}
    &a \equiv b \ (mod \ m)\\
    \iff& m \mid (a - b)\tag{by definition}\\
    \iff& a - b = km\tag{$k \in \ints$}\\
    \iff& b - a = -km\\
    \iff& m \mid (b -a)\\
    \iff& b \equiv a \ (mod \ b)\tag{by definition}
\end{align*}
\begin{theorem}\label{theorem4.1.3}
    Let $a,b,c,d,m \in \ints$ with $m \geq 2$. If $a \equiv b \ (mod \ m)$ and $c \equiv d \ (mod \ m)$, then 
    \begin{enumerate}
        \item $a + c \equiv b + d \ (mod \ m)$
        \item $ac \equiv bd \ (mod \ m)$
    \end{enumerate}
\end{theorem}
    \begin{proof}
    \begin{enumerate}
        \item We have to prove $a + c \equiv b + d \ (mod \ m)$. Since $a \equiv b$ and $c \equiv d$, we have 
        $$m \mid (a - b)$$
        $$m \mid (c - c)$$
        By theorem 4.1.1 (part 1),we have 
        $$m \mid ((a-b) + (c -d)$$
        $$m \mid ((a+c) - (b + d))$$
        Therefore,
        $$a + c \equiv b + d \ (mod \ m)$$
        \item We have to prove $ac \equiv cd \ (mod \ m)$\\[1ex]
        Since $a \equiv b$ and $c \equiv d$, we have $m \mid (a-b)$ and $m \mid (c -d)$. By \hyperref[corollary4.1.1]{Corollary 4.1.1}, we have 
        $$m \mid (c(a-b)+b(c-d)$$
        $$m \mid (ac - bc + bc- bd)$$
        $$m \mid (ac - bd)$$
        Therefore $ac \equiv bd$.
    \end{enumerate}
    \end{proof}

    \section{Arithmetic Modulo m}
    Let $m \geq 2$ be an integer and 
    $$\ints_m = \{0,1,2,\dots, m-1\}$$
    We define 
    $$a +_m b = (a + b) \ (mod \ m)$$
    $$a \cdot_m b = (a \cdot b) \ (mod \ m)$$
    in $\ints_m$, this is arithmetic modulo m.
    TBC


%%%%%%%%%%%%%%%%%% Lecture 5 %%%%%%%%%%%%%%%%%%%%%%%%%%%%%%%%%
\chapter{Prime Numbers and GCD}
\section{Prime Numbers}
\begin{definition}\label{definition5.1.1}
    A positive integer $p$ is prime if it admits exactly two divisors.
\end{definition}
\begin{theorem}[Fundamental Theorem of Arithmetic]\label{theorem5.1.1}
    All integers greater than 1 can be written as a product of prime numbers. This representation is unique if we write the prime numbers in non-decreasing order.
\end{theorem}
\begin{proof}
\textbf{(Existence)} By induction,
    \begin{itemize}
        \item \textbf{Base Case:} Take $n = 2$. We have $2 = 2$, the product of 1 prime number. 
        \item \textbf{Induction Hypothesis:} Let $k \geq 2$ be an integer. Suppose that all numbers $2,3,4, \ldots, k-1, k$ can be written as a product of primes.
        \item \textbf{Induction Step:} Consider $k +1$. If $k+1$ is prime, then we're done. If not, then $K+1 = d\cdot e$ for integers $1 < d < k+1$ and $1 < e < k+1$
        By the induction hypothesis, $d$ and $e$ can be written as products of prime. So $k + 1 = d \cdot e$ can be written as a product of primes. 
    \end{itemize}
\textbf{(Uniqueness)} to be seen later.
\end{proof}
\begin{theorem}\label{theorem5.1.2}
    Let $n > 1$ be an integer. If $n$ is not prime, then $n$ has a prime divisor $p$ such that $p \leq \sqrt{n}$.
\end{theorem}
\begin{proof}
    Let $n > 1$, if $n$ is not prime, then $n = a \cdot b$ for two integers $1 < a < n$ and $1 < b < n$. We will show that $a \leq \sqrt{n}$ or $b \leq \sqrt{n}$ by contradiction. Assume $a > \sqrt{n}$ and $b > \sqrt{n}$. Then $n = a \cdot b > \sqrt{n} \cdot \sqrt{n} = n$. This is a contradiction so $a \leq \sqrt{n}$. \\[1ex] 
    Assume without loss of generality that $a \leq \sqrt{n}$. If $a$ is prime, we're done. If not, then by the fundamental theorem of arithmetic, $a$ is divisible by a prime number $p$
\end{proof}
\begin{theorem}\label{theorem5.1.3}
There exists an infinite number of prime numbers.    
\end{theorem}
\begin{proof}
    By contradiction, suppose there exists a finite number of prime numbers, say $k$ prime numbers, and we order them 
    $$p_1 < p_2 < p_3 < \cdots < p_k$$
    Consider the number 
    $$Q = p_1 \cdot p_2 \cdot \ldots \cdot p_k + 1 \in \ints$$
    Since $Q > p_k$, then $Q$ is not prime by our assumption. By \hyperref[theorem5.1.2]{Theorem 5.1.2}, Q is divisible by a prime number. So $p_i \mid Q$ for some $1 \leq i \leq k$. We also have that 
    $$p_i \mid (p_1\cdot p_2 \cdot \ldots \cdot p_i \cdot \ldots \cdot p_k)$$
    By \hyperref[corollary4.1.1]{Corollary 4.1.1}, we get
    $$p_i \mid (Q - p_1 \cdot p_2 \cdot \ldots \cdot p_k)$$
    $p_i \mid 1$
    Therefore $p_i = 1$, this is a contradiction since we assumed $p_k$ is the largest prime but $Q > p_k$ is prime. 
\end{proof}
%%%%%%%%%%%%%%%%%% Lecture 6 %%%%%%%%%%%%%%%%%%%%%%%%%%%%%%%%%
\chapter{Euclidean Algoirthm and B\'ezout's Theorem}
\begin{corollary}\label{corollary6.0.1}
    Let 
    $$a = p_1^{a_1} \cdot p_2^{a_2} \cdot \ldots \cdot p_k^{a_k}$$
    $$a = p_1^{b_1} \cdot p_2^{b_2} \cdot \ldots \cdot p_k^{b_k}$$
    Where $p_i$ is prime, $a_i \geq 0$ and $b_i \geq 0$, $1 \leq i \leq k$. Then 
    $$gcd(a,b) = p_1^{\min(a_1,b_1)} \cdot p_2^{\min(a_2,b_2)} \cdot \ldots \cdot p_k^{\min(a_k,b_k)}$$
    $$lcm(a,b) = p_1^{\max(a_1,b_1)} \cdot p_2^{\max(a_2,b_2)} \cdot \ldots \cdot p_k^{\max(a_k,b_k)}$$
    $$gcd(a,b) \cdot lcm(a,b) = ab$$
\end{corollary}
\textbf{Example:} 
\begin{align*}
    24 &= 2^3 \cdot 3\\
    36 &= 2^2 \cdot 3^2\\
    gcd(24,36) &= 2^2 \cdot 3^1 = 12\\
    lcm(24,36) &= 2^3 \cdot 3^2 = 72\\
    12 \cdot 72 &= 864 = 24 \cdot 36
\end{align*}
\begin{lemma}\label{lemma6.0.1}
    Let $a,b,q,r$ be integers such that 
    $$a = b\cdot q + r$$
    Then 
    $$gcd(a,b)=gcd(b,r)$$
\end{lemma}
\begin{proof}
     Let $a,b,q,r$ be integers such that 
     $$a = bq + r$$
     Let $d \in \ints$. We will prove that 
    $$d \mid a \wedge d \mid b \iff d \mid b \wedge d\mid r$$
    ($\implies$) Let $d \in \ints$. Assume $d \mid a$ and $d \mid b$. Then $d \mid (1 \cdot a + (-q)\cdot b)$, by \hyperref[corollary4.1.1]{Corollary 4.1.1}. Then $a = bq + r \implies r = a -bq$, so $d \mid (1 \cdot a + (-q)\cdot b) \implies d \mid r$.\\[2ex]
    ($\impliedby$) Let $d \in \ints$. Assume $d \mid b$ and $d \mid r$. Then $d \mid (q\cdot b + 1 \cdot r)$ by \hyperref[corollary4.1.1]{Corollary 4.1.1}. Then $d \mid a$, therefore $d \mid a$ and $d \mid b$
\end{proof}
\textbf{Example:} $gcd(414,662)$, $662 = 1 \cdot 414 + 248$
\begin{align*}
    662 &= 1 \cdot 414 + 248\\
    414 &= 1 \cdot 248 + 166\\
    248 &= 1 \cdot 166 + 82\\
    166 &= 2 \cdot 82 + 2 \\
    82 &= 41 \cdot 2 + 0
\end{align*}
The last none-zero remainder of this sequence is the $gcd$ of 414 and 662 by the previous lemma. (can someone find which lemma this is!)
\begin{definition}[Euclidean Algorithm]\label{definition6.0.1}
    $$x = a$$
    $$y = b$$
    while $y \neq 0$
    $$r = x \mod y$$
    $$x = y$$
    $$y = r$$
    return $x$
\end{definition}
This algorithm returns the $gcd$ of $a$ and $b$.\\[3ex]
\textbf{Example:}$gcd(465,144)$
\begin{align*}
    465 &= 3 \cdot 144 + 33\\
    144 &= 4 \cdot 33 + 12\\
    33 &= 2 \cdot 12  + 9\\
    12 &= 1 \cdot 9 + 3 \\
    9 &= 3 \cdot 3 + 0
\end{align*}
Therefore $gcd(465,144) = 3$. 
\begin{center}
    \red{\textbf{Note: When you show the trace of Euclid's algorithm, you must include the last line with a remainder of 0.}}
\end{center}
\begin{theorem}[B\'ezout]\label{theorem6.0.1}
    Let $a,b \in \ints$ be positive integers. There exists $s,t \in \ints$ such that 
    $$s \cdot a + t \cdot b = gcd(a,b)$$
\end{theorem}
\begin{proof}
    Let $a,b \in \nat \setminus \{0\}$. Run Euclidian algorithm, and assume without loss of generality $b \leq a$.
    \begin{align*}
        a &= q \cdot b + r\\
        r_0 &= q_1 \cdot r_1 + r_2\\
        r_1 &= q_2 \cdot r^2 + r_3\\
        r_2 &= q_3 \cdot r_3 + r_4\\
        \vdots \\
        r_{n-3} &= q_{n-2} \cdot r_{n-2} + r_{n-1}\\
        r_{n-2} &= q_{n-1} \cdot r_{n-1} + r_{n}\\
        r_{n-1} &= q_{n} \cdot r_{n} + 0\\
    \end{align*}
    Then, we have 
    \begin{align*}
        gcd(a,b) &= r_n\\
        &= r_{n-2} - q_{n-1} \cdot r_{n-1}\\
        &= r_{n-2} - q_{n-1}(r_{n-3} - q_{n-2}r_{n-2})\\
        &= r_{n-2} - q_{n-1}(r_{n-3} - q_{n-2}r_{n-2})\\
        &= - q_{n-1}\cdot + (1 + q_{n-2}q_{n-1})\cdot r_{n-2}\\
        \vdots \\
        &= s \cdot r_0 + t \cdot r_1\\
        &= s \cdot a + t \cdot tb
    \end{align*}
    So we read the trace of Euclid's algorithm backward while keeping $gcd(a,b)$ on the same side of the equality.
\end{proof}
\textbf{TBC.}
\begin{lemma}\label{lemma6.0.2}
    Let $a,b,c \in \ints$ with $a \neq 0$. If $gcd(a,b) = 1$ and $a \mid (bc)$, then $a \mid c$.
\end{lemma}
\begin{proof}
    Assume $gcd(a,b) = 1$ and $a \mid (bc)$. By B\'ezout, there exist $s,t \in \ints$ such that 
    \begin{align*}
        s \cdot a + t \cdot b &= gcd(a,b) = 1\\
        s \cdot a \cdot c + t \cdot b \cdot c &= c\tag{*}
    \end{align*}
    Since $a \mid a$ and $a \mid (bc)$, we have 
    $$a \mid (s \cdot c \cdot a + t \cdot b \cdot c$$
    By \hyperref[corollary4.1.1]{Corollary 4.1.1}. Then from (*), this means 
    $$a \mid c$$
\end{proof}
%%%%%%%%%%%%%%%%%% Lecture 7 %%%%%%%%%%%%%%%%%%%%%%%%%%%%%%%%%
\chapter{Applications of B\'ezout's Theorem}
TBC. 

\chapter{GCD and Modulo n, Multiplicative Inverses in Modulo n}
\begin{lemma}\label{lemma8.0.1}
Let $a,b,c \in \ints$, with $a \neq 0$. If $gcd(a,b) = 1$, and $a \mid (bc)$, then $a|c$.
\end{lemma}
\begin{proof}
    Seen last week.
\end{proof}
\begin{theorem}\label{theorem8.0.1}
    Let $a,b,c,m \in \ints$, with $m \geq 2$. Assume $ac \equiv bc \ (mod \ m)$ and $gcd(c,m) = 1$. Then $a \equiv b \ (mod \ m)$.
\end{theorem}
\begin{proof}
    Let $a,b,c,m \in \ints$ with $m \geq 2$. Assume $ac \equiv bc \ (mod \ m)$ and $gcd(c,m) = 1$.
    \begin{align*}
        m \mid (ac-bc)\tag{def of mod}\\
        m \mid (c(a-b))\\
        m \mid (a-b)\tag{by previous lemma}\\
        a \equiv b \ (mod \ m)\tag{def of mod}
    \end{align*}
\end{proof}
\begin{lemma}\label{lemma8.0.2}
    Let $p$ be a prime number and $a_1, a_2,\dots,a_n \in \ints$.
    If $p \mid (a_1 \cdot a_2 \cdot \dots \cdot a_n)$, then there exists $1 \leq i \leq n$ such that $p \mid a_i$.
\end{lemma}
\begin{proof}
    By induction on $n$.
    \begin{itemize}
        \item \textbf{Base Case:} $n = 1$. Let $p$ be a prime number, if $p \mid a_1$, then $p\mid a_1$
        \item \textbf{Induction Hypothesis:} Let $k \geq 1$ be an integer. Suppose that for all integers $a_1, a_2, \dots, a_k$
        $$p\mid(a_1 \cdot a_2 \cdot \ldots \cdot a_k) \implies \exists 1 \leq i \leq k \ s.t \ p\mid a_i$$
        If $p \mid a_{k+1}$, then we're done. If not, then 
        $$gcd(p,a_{k+1}) = 1$$
        So $p\mid(a_1 \cdot a_2 \cdot \ldots \cdot a_k)$ by the previous lemma. By the induction hypthesis, ther exists $1 \leq i \leq k$ such that $p \mid a_i$.
    \end{itemize}
    \item \textbf{Induction Step:} Suppose
    $$p \mid (a_1 \cdot a_2 \cdot \ldots \cdot a_k \cdot a_{k+1})$$
\end{proof}
\begin{theorem}\label{theorem8.0.2}
    Let $m \in \ints$ with $m \geq 2$ and let $a \in \ints_m$. The multiplicative inverse of $a \ (mod \ n)$ exists if and only if $gcd(a,m) = 1$. When it exists, the inverse of  $a \ (mod \ n)$ is unique.
\end{theorem}
\begin{proof}
    Let $m\in \ints$ with $m \geq 2$ and $a \in \ints_m$\\[2ex]
    \textbf{($\implies$):} Assume the multiplicative inverse of  $a \ (mod \ n)$ exists. Let $\bar{a}$ be this inverse. By definition,
    \begin{align*}
        a \cdot \bar{a} &\equiv 1 \ (mod \ m)\\
        m &\mid (a \cdot \bar{a} - 1)\tag{def. of modulo}
    \end{align*}
    Then, $a \cdot \bar{a} - 1 = k \cdot m$ for some $m \in \ints$. Let $d = gcd(a,m)$ Then $d|a$ and $d|m$. By a result seen in class, 
    \begin{align*}
        d &\mid (\bar{a} \cdot a + (-k)m)\\
        d &\mid 1
    \end{align*}
    So, $d = 1$\\[2ex]
    \textbf{($\impliedby$):} Assume $gcd(a,m) = 1$. By B\'ezout, there exists $s,t \in \ints$ such that 
    $$s\cdot a + t \cdot m = gcd(a,m) = 1$$
    \begin{align*}
        s \cdot a + t \cdot m \equiv 1 \ (mod \ m)\\
        s \cdot a + t \cdot 0 \equiv 1\ (mod \ m)\\
        s \cdot a \equiv 1 \ (mod \ m)\\
    \end{align*}
    So, we can take $\bar{a} \equiv s \ (mod \ m)$\\[2ex]
    \textbf{(Uniqueness):} Consider two arbitrary multiplicative inverses of $a \ (mod \ m)$. Denote them by, $s, s' \in \ints_m$. So by definition
    $$sa \equiv 1 \ (mod \ m) \text{ and } s'a \equiv 1 \ (mod \ m)$$
    Then $gcd(a,m) = 1$ by the previous proof, also we have 
    \begin{align*}
        sa &\equiv s'a \ (mod \ m)\\
        m &\mid (sa - s'a)\tag{def. of modulo}\\
        m &\mid (a(s-s'))\\
        m &\mid (s-s')\tag{since $gcd(a,m)=1$}\\
        s &\equiv s' \ (mod \ m)\tag{def. of modulo}
    \end{align*}
    Therefore, $s$ and $s'$ are the same in $\ints_m$.
\end{proof}
\textbf{Example:} Find the multiplicative inverse of 101 (mod 4620).\\[2ex]
\textbf{Euclid:}
\begin{align*}
    4620 & = 45 \cdot 101 + 75\\
    101 &= 1 \cdot 75 + 26\\
    75 &= 2 \cdot 26 + 23\\
    26 &= 1 \cdot 23 + 3\\
    23 &= 7 \cdot 3 + 2\\
    3 &= 1 \cdot 2 + 1\\
    2 &= 2\cdot1 + 0
\end{align*}
\textbf{B\'ezout:}
\begin{align*}
    1 &= 3 - 1\cdot 2\\
    1 & = 3 - 1 \cdot (23 -  7 \cdot 3)\\
    1 & = 3 - 1\cdot 23 + 7\cdot 3\\
    1 &= 8\cdot 3  - 1\cdot 23\\
    1 &= -1 \cdot 23 + 8 \cdot 3\\
    1 &= -1 \cdot 23 + 8 \cdot (26 - 1 \cdot 23)\\
    1 &= -1 \cdot 23 + 8 \cdot 26 - 8 \cdot 23\\
    1 &= -9 \cdot 23 + 8\cdot 26\\
    1 &= 8 \cdot 26 - 9 \cdot 23\\
    1 &= 8 \cdot 26 - 9 \cdot (75 - 2 \cdot 26)\\
    1 & = 8 \cdot 26 - 9\cdot 75 + 18 \cdot 26\\
    1 &= -9 \cdot 75 + 26 \cdot 26\\
    1 &= -9 \cdot 75 + 26 \cdot (101 - 1 \cdot 75)\\
    1 &= -9 \cdot 75 + 26 \cdot 101 -26 \cdot 75\\
    1 &= 26 \cdot 101 - 35 \cdot 75\\
    1 &= 26 \cdot 101 - 35 \cdot (4620 - 45\cdot 101)\\
    1 &= 26\cdot 101 - 35 \cdot 4620 + 1575\cdot 21\\
    1 &= -35 \cdot 4620 + 1601 \cdot 101
\end{align*}
So,
\begin{align*}
    -35 \cdot 4620 + 1601 \cdot 101 &\equiv 1 \ (mod \ 4620)\\
    -35 \cdot 0 + 1601 \cdot 101 &\equiv 1 \ (mod \ 4620)\\
    1601 \cdot 101 &\equiv 1 \ (mod \ 4620)\\
    101 &\equiv 1601 \ (mod \ 4620)
\end{align*}
Therefore, the inverse of 101 in $\ints_{4620}$  is 1601. \\[3ex]
\textbf{Example:} Find the multiplicative inverses in $\ints_{10}$.
\begin{itemize}
    \item $\bar{0}$ does not exist since $gcd(0,10) = 10 \neq 1$
    \item $\bar{1} \equiv 1 \ (mod \ 10)$
    \item $\bar{2}$ does not exist since $gcd(2,10) = 2 \neq 1$
    \item $\bar{3} \equiv 7 \ (mod \ 10)$
    \item $\bar{4}$ does not exist since $gcd(4,10) = 2 \neq 1$
    \item $\bar{5}$ does not exist since $gcd(5,10) = 5 \neq 1$
    \item $\bar{6}$ does not exist since $gcd(6,10) = 2 \neq 1$
    \item $\bar{7} \equiv 3 \ (mod \ 10)$
    \item $\bar{8}$ does not exist since $gcd(8,10) = 2 \neq 1$
    \item $\bar{9} \equiv 9 \ (mod \ 10)$
\end{itemize}
\begin{center}
    \textbf{This concludes the material for midterm 1.}
\end{center}
\chapter{Solving Congruences}
\begin{definition}[Linear Congruence]
$ax \equiv b$ (mod $m$)
\end{definition}
\textbf{Example:}
$$3x \equiv 5 \ (mod \ 7)$$
$$x \equiv 0 \ (mod \ 7)$$
$$x - 0 = 7k$$
\textbf{Question:} What is the multiplicative inverse of $3$ (mod $7$)
So we have $3x \equiv 5$ (mod $7$). 
$$15x \equiv 25 \ (mod \ 7)$$
$$x \equiv 4 \ (mod \ 7)$$
$$3 \cdot 4 = 12 \equiv 5 \ (mod \ 7)$$

\section{Linear Congruence System}
Find $x$ such that
$$x \equiv a_1 \ (mod \ m_1)$$
$$x \equiv a_2 \ (mod \ m_n)$$
$$\vdots$$
$$x \equiv a_n \ (mod \ m_n)$$

\textbf{Example:}
$$x \equiv 2\ (mod \ 3)$$
$$x \equiv 3\ (mod \ 5)$$
$$x \equiv 5\ (mod \ 7)$$
Try $x = 68$
$$68 \equiv 2 \ (mod \ 3)$$
$$68 \equiv 3 \ (mod \ 5)$$
$$68 \equiv 5 \ (mod \ 7)$$
So, $x = 68$ is a solution to the system.
\subsection{Substitution Method}
$$x \equiv 2 \ (mod \ 3)$$
$$x = 3 \cdot t + 2$$
For some $t \in \ints$
$$x \equiv \ (mod \ 5)$$
$$3t + 2 \equiv 3 \ (mod \ 5)$$
$$3t \equiv 1 \ (mod \ 5)$$
Multiply $3x$ by the multiplicative inverse of $3$ in $\ints_5$.
$$2 \cdot 3t \equiv 2 \cdot 1 \ (mod \ 5)$$
$$t \equiv 2 \ (mod \ 5)$$
$$t = 5u + 2 \ (mod \ 5)$$ 
For an $u \in \ints$\\
\begin{center}
    \[\begin{rcases}
$$x = 3t + 2$$\\
$$t = 5u + 2$$
\end{rcases}\] $\implies x =$ ?
\end{center}
$$x = 3(5u +2) +2 = 15u + 8$$

\begin{align*}
15u + 8 &\equiv 5 \ (mod \ 7)\\
15u &\equiv -3 \ (mod \ 7)\\ 
15u &\equiv 4 \ (mod \ 7) \\
15u - 14u &\equiv 4 \ (mod \ 7)\\
u &\equiv 4 \ (mod \ 7)
\end{align*}
So $u = 7v + 4$ for some $v \in \ints$. Thus,
\begin{align*}
    x &= 15u + 8\\
    &= 15(7v + 4) + 8\\
    &= 105v + 68
\end{align*}
So,
\begin{align*}
    105v + 68 &\equiv 2 \ (mod \ 3)\\
    105v + 68 &\equiv 3\ (mod \ 5)\\
    105v + 68 &\equiv 5\ (mod \ 7)\\
\end{align*}

\noindent
\textbf{Example:}
\begin{align*}
    x &\equiv 1 \ (mod \ 4)\\
    x &\equiv 3 \ (mod \ 5)
\end{align*}
Then $x = 4t + 1$ for some $t \in \ints$. Then from the second equation, we get 
\begin{align*}
    4t + 1 &\equiv 3 \ (mod \ 5)\\
    4t + 1 - 1 &\equiv 3 -1 \ (mod \ 5)\\
    4t  &\equiv 2 \ (mod \ 5)\\
    4\cdot4t  &\equiv 4\cdot2 \ (mod \ 5)\\
    16t  &\equiv 8 \ (mod \ 5)\\
    16t  &\equiv 8 \ (mod \ 5)\\
    16t - 15t  &\equiv 8 - 5\ (mod \ 5)\\
    t  &\equiv 3\ (mod \ 5)\\
\end{align*}
Thus, $t = 5u + 3$ for some $u \in \ints$. So $x = 20u + 13$ is a solution to the system. 
\begin{align*}
    20u + 13 &\equiv 1 \ (mod \ 4)\\
    20u + 13 &\equiv 3 \ (mod \ 5)\\
\end{align*}

\noindent
\textbf{Question:} Are there systems that admit no solution? Consider
$$x \equiv 2 \ (mod \ 4)$$
$$x \equiv 3 \ (mod \ 6)$$
So $x = 4t + 2$ for some $t \in \ints$
$$4t + 2 \equiv 3 \ (mod \ 6)$$
$$4t \equiv 1 \ (mod \ 6)$$
But, 4 does not have a multiplicative inverse in $\ints_6$ since $gcd(4,6)\neq 1$. 
\begin{theorem}[Chinese Remainder Theorem]
Let $m_1, m_2, \dots, m_r \in \ints$ be pairwise co-prime integers such that $m_i \geq 2$ for $1 \leq i \leq r$
\end{theorem}
\begin{definition}[Pairwise Co-prime]
    $gcd(m_i,m_j) = 1$
\end{definition}
Let $a_1, a_2, \dots, a_r \in \ints$, then the system
\begin{align*}
    x &\equiv a_1 \ (mod \ m_1)\\
    x &\equiv a_2 \ (mod \ m_2)\\
    &\vdots\\
    x &\equiv a_r \ (mod \ m_r)
\end{align*}
admits a unique solution $(mod \ m_1 \cdot m_2 \cdots m_r)$.In other words, the solution exists and is unique in $\ints_{m_1 \cdot m_2 \cdots m_r}$\\[2ex]
Consider the system
\begin{align*}
    x \equiv 2 \ (mod \ 3)\\
    x \equiv 3 \ (mod \ 5)\\
    x \equiv 5 \ (mod \ 7)\\
\end{align*}
So we have $\ints_{3 \cdot 5 \cdot 7} = \ints_{105}$, $68 \in \ints_{105}$ and $x = 105u + 68$. 

\chapter{Fermat's Theorem}
\begin{theorem}[Fermat's Theorem]
    Let $p,a \in \ints$ such that $p$ is prime, then 
    \begin{enumerate}
        \item
    $$a^p \equiv a \ (mod \ p)$$
        \item
    If $gcd(a,p) = 1$, then $a^{p-1} \equiv 1 \ (mod \ p)$
        
    \end{enumerate}
\end{theorem}
\textbf{Example:} 
$$1534^{2016} \ (mod \ 2017)$$
2017 is prime and $1534 < 2017$, so $gcd(1534, 2017) = 1$ and $1534^2016 \equiv 1 \ (\mod \ 2017)$
\begin{proof}For (2)
$1 \cdot a, 2\cdot a, 3 \cdot a, \dots, (p-1)\cdot a$ are all different $(mod \ p)$. 
\raggedcolumns
\begin{multicols}{2}
\begin{itemize}
    \item $a = 12$ and $p = 7$
    \item $1 \cdot a \equiv 5 \ (mod \ 7)$
    \item $2 \cdot a \equiv 3 \ (mod \ 7)$
    \item $3 \cdot a \equiv 1 \ (mod \ 7)$
    \item $4 \cdot a \equiv 6 \ (mod \ 7)$
    \item $5 \cdot a \equiv 4 \ (mod \ 7)$
    \item $6 \cdot a \equiv 2 \ (mod \ 7)$
\end{itemize}
\columnbreak
\begin{itemize}
    \item $a = 9$ and $p = 5$
    \item $1 \cdot a \equiv 4 \ (mod \ 5)$
    \item $2 \cdot a \equiv 4 \ (mod \ 5)$
    \item $3 \cdot a \equiv 4 \ (mod \ 5)$
    \item $4 \cdot a \equiv 4 \ (mod \ 5)$
\end{itemize}
\end{multicols}



\end{proof}

%------------ Skip Lecture 11 ------------
\setcounter{chapter}{11}

%----------------- Lecture 12 -----------------
\chapter{Intro to Cryptography}

%----------------- Lecture 13 -----------------
\chapter{Asympotic Notation}

\section{Big-O Notation}
The \emph{O-notation} describes an asympotic upper bound.
\begin{definition}
    Let 
    \[f: \nat \rightarrow \real^+\]
    \[g: \nat \rightarrow \real^+\]
    be two functions. We say that $f$ is \emph{O(g)} if there exists a real number $c > 0$ and $k \in \nat$ such that for all $n \geq k$, 
    \[f(n) \leq c \cdot g(n)\]
\end{definition} 
\noindent
\textbf{Notation:}
\[f(n) \leq c \cdot g(n)\]
\[f = O(g)\]
\[\exists c\exists k \forall n (n \geq k \implies f(n) \leq c \cdot g(n))\]
\textbf{Domain:} $k, n \in \nat$, $c \in \real^+ \setminus \{0\}$\\[3ex]
\noindent
\textbf{Example:} $13x^3 + 12x^2 + 5 = O(x^3)$. We have
\[13x^3 + 12x^2 + 5 \leq 13x^3 + 12x^3 + 5x^2 = 30x^3\]
Take $c = 30$ and $k = 1$. So 
\[13x^3 + 12x^2 + 5 \leq 30 \cdot x^3 \]
for all $x \geq 1$. Therefore $13x^3 + 12x^2 + 5 = O(x^3)$. \\[3ex]
\textbf{Example:} $x^2 = O\left(\frac{1}{2}x^2 - 10x\right)$. We have 
\[x^2 \leq 2 \left(\frac{1}{2}x^2 - 10x\right)\]
Now we want 
\[x^2 \geq 40x\]
so that that $x^2 - 40x$ is positive. So
\[x > 40\]
Then, 
\begin{align*}
    x^2 &= 2x^2 - x^2\\
    &\leq 2x^2 - 40x\\
    &= 4\left(\frac{1}{2}x^2 - 10x\right)
\end{align*}
So take $c = 4$ and $k = 40$. Then $x^2 = O\left(\frac{1}{2}x^2 - 10x\right)$ for all $x \geq 40$.


\begin{prop}
    Let $a > 0$ and $b > 0$. be two rael numbers. We have 
    \[log^a(x) = O(x^b)\]
\end{prop}
\begin{proof}
    Let $a > 0$ and $b > 0$ be two real numbers. We'll use that fact that $\forall x \geq 0$, we have $x \leq e^x$. From which, we have $log(x) \leq x$. Let $x$ be an integer. We have, by the previous property, 
    \begin{align*}
        \log(x^{\frac{b}{a}}) &\leq x^{\frac{b}{a}}\\
        \frac{b}{a}\log(x) &\leq x^{\frac{b}{a}}\\
        \left(\frac{b}{a}\right)^a \log^a(x) &\leq x^b\\
        \log^a(x) &\leq \left(\frac{a}{b}\right)^ax^b
    \end{align*}
    So we take $c = \left(\frac{a}{b}\right)^a$ and $k = 1$.
\end{proof}

\section{Big-Omega Notation}
The \emph{$\Omega$-notation} describes an asympotic lower bound.
\begin{definition}
    Let 
    \[f: \nat \rightarrow \real^+\]
    \[g: \nat \rightarrow \real^+\]
    be two functions. We say that $f$ is $\Omega(g)$ if there exists a real number $c > 0$ and $k \in \nat$ such that for all $n \geq k$,
    \[f(n) \geq c\cdot g(n)\]
\end{definition}
\noindent
\textbf{Notation:} 
\[f(n) = \Omega(g(n))\]
\[f = \Omega(g)\]
\[\exists c\exists k \forall n (n \geq k \implies f(n) \geq c \cdot g(n))\]
\textbf{Domain:} $k, n \in \nat$, $c \in \real^+ \setminus \{0\}$\\[3ex]
\textbf{Example:} $13x^3 + 12x^2 + 5 = \Omega(x^3)$.
\[13x^3 + 12x^2 + 5 \geq 13x^3\]
Take $c = 13$ and $k = 0$. So $13x^3 + 12x^2 + 5 = \Omega(x^3)$.\\[3ex]
\textbf{Example:} $x^2 = \Omega\left(\frac{1}{2}x^2 - 10x\right)$. 
\begin{align*}
    x^2 &\geq \frac{1}{2}x^2\\
    &\geq \frac{1}{2}x^2 - 10x\\
    &= 1 \cdot \left(\frac{1}{2}x^2 - 10x\right)
\end{align*}
Take $c = 1$ and $k = 0$. So $x^2 = \Omega\left(\frac{1}{2}x^2 - 10x\right)$ $\forall x \geq k$.
\begin{prop}
    Let $f(n)$ and $g(n)$ be two functions. 
    \[f(n) = O(g(n)) \iff g(n) - \Omega(f(n))\]
\end{prop}
\begin{proof}
    ($\implies$) Let $f(n)$ and $g(n)$ be two functions. Assume $f(n) = O(g(n))$. Then there exists $c > 0$ and $k \in \nat$ such that for all $n \geq k$, we have $f(n) \leq c \cdot g(n)$. So,
    \[f(n) \leq c \cdot g(n)\]
    given that $n \geq g(n)$, then 
    \[g(n) \geq \frac{1}{c} f(n)\]
    ($\impliedby$) The proof follows the same.
\end{proof}

\section{Big-Theta Notation}
The \emph{$\Theta$-notation} describes an asympotic upper and lower bound.
\begin{definition}
    Let 
    \[f: \nat \rightarrow \real^+\]
    \[g: \nat \rightarrow \real^+\]
    be two functions. We say that $f$ is $\Theta(g)$ if there exists a real number $c_1 > 0$, $c_2 > 0$ and $k \in \nat$ such that for all $n \geq k$. In otherwords,
    \[f(n) = O(g(n)) \text{ and } f(n) = \Omega(g(n))\]
\end{definition}
\textbf{Notation:} 
\[f(n) = \Theta(g(n))\]
\[f = \Theta(g)\]
\begin{prop}
    Let $f(n)$ and $g(n)$ be two functions. $f(n) = \Theta(g(n))$ if and only if $f(n) = O(g(n))$ and $g(n) = \Omega(f(n))$.
\end{prop}
\begin{proof}
    \[f(n) = \Theta(g(n)) \iff f(n) = O(g(n)) \text{ and } f(n) = \Omega(g(n))\]
    By the definition of theta, so
    \[g(n) = \Omega(f(n)) \text{ and } g(n) = O(f(n))\]
    From the previous proposition, then 
    \[g(n) = \Theta(f(n))\]
\end{proof}
\end{document}

 